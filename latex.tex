\documentclass[12pt]{article}
\usepackage{mgates-letter}
\definecolor{dark_blue} {rgb}{0., 0., 0.65}
\usepackage{makecell}
\usepackage[
backend=biber,
style=ieee,
%citestyle=numeric,
%entrykey=false,
%annotation=false,
url=false
]{biblatex}

\renewbibmacro*{finentry}{%
  \setunit{\finentrypunct\par}%
	\usebibmacro{abstract}%\
\finentry}
\usepackage{textcomp}
\usepackage{mathrsfs}  % mathscr font
\usepackage{boxedminipage}
\usepackage{rotating}
\usepackage{csquotes}
%\usepackage{natbib}
\usepackage[colorlinks, filecolor=dark_blue, urlcolor=dark_blue, linkcolor=black, citecolor=black]{hyperref}

\defbibcheck{mine}{\iffieldequalstr{annotation}{mine}{}{\skipentry}}
\defbibcheck{other}{\iffieldequalstr{annotation}{other}{}{\skipentry}}

\addbibresource{biblio.bib}
\begin{document}

\title{PhD end-of-year report, I}
\author{Gianluca Aguzzi, PhD student XXX cycle}
\date{\today}
\maketitle

\section{Overall activity}
My current research topic concerns engineering large scale multi-agent systems. In particular, we are
 focused on \textit{Cyber Physical Swarms}(CPS), a broad term that refers to a ``swarm" (like whom
 existing in nature) of computational (simple) entities that, through emergent behaviour, reach goals collectively.  
%
Here, engineers have to deal with distributed control, huge nodes count, openness, highly stochastic environments.
%
So in literature, different solutions aim at handling this complexity but, we claim that Aggregate Computing is one of the most
 prominent approach to ``program" collective behaviour for such a kind of system.
%
Indeed it enables the definition of self-organising collective behaviour through a distributed data manipulation called \textit{computational field}. In this way, the program specification is ``collectively" declare, so, theoretically,
 can scale up at any system size.
%
Aggregate Computing is successfully applied in various scenarios, such as crowds of people, swarms of robots and smart cities.
%
Anyway, Aggregate Computing needs to widen both at the foundational level and at the "practical level".

By the foundational level, In this first year, my activities are focused on the integration of field-based coordination paradigms 
 -- In particular Aggregate Computing -- with Machine Learning capability 
 in order to create even more intelligent collective behaviour.
%
Indeed, even if Aggregate Computing has already been applied in various scenarios -- 
 such as crowds of people, swarms of robots and smart cities -- the behaviour specification can lead to
 a complex fine-tuning guided sometimes by constant definition derived by domain experience.
%
Furthermore, some behaviour works well only in specific contexts and when environmental
 changes happen the behaviour became unstable.
%
So we think that learning can lead to better collective behaviours, 
 improving our current state-of-the-art solutions, making Aggregate 
 Computing is more and more applicable in real large scale systems.
%
In the first part of my work, I tried to understand what 
 kind of machine learning algorithm is most suitable for Aggregate Computing.
%
My works climax with \textit{\citetitle{research}} where I explain the different suitable approach
 to enhance Aggregate Computing with Machine learning.
%
Finally, in the last period, I made concrete experiments with 
 Reinforcement Learning -- in particular Independent Q-Learning -- improving the current Aggregate Computing solution. 
 These works do not already be described in a publication,
 we count on doing this in few months.

Aside to ``Machine Learning'' integration, in \textit{\citetitle{collectiveautonomy}} we explore the relationship
 between Aggregate Computing and agents autonomy. In particular, we investigate the position of Aggregate Computing 
 w.r.t individual autonomy and the collective autonomy describing how we can selectively adjust" it. 

Speaking about the "practical" level, my research activities aim at closing the gap between
 its abstract space and its application in concrete systems. 
%
In this direction, I made mainly contributions to ScaFi~\cite{scafi}, 
 a research-oriented tool-chain that brings Aggregate Computing into the Scala world. The works are done consist in:

\begin{itemize}
  \item \textit{\citetitle{scafiweb}}: 
	A playground to experiment with Aggregate Computing. From a long-term perspective, this tool should be used in support of
  the distributed monitoring of Aggregate Computing application
  \item \textit{\citetitle{scafiloci}}: 
	An approach for deploying aggregate programs in different networks. This work have a long-term vision of separating functional aspects from non-functional aspects.
\end{itemize}

\section{International Conferences Activities}
\begin{enumerate}
	\item Talk at \href{https://www.discotec.org/2021/programme}{COORDINATION 2021} for \cite{scafiweb}
	\item Talk at \href{https://apice.unibo.it/xwiki/bin/view/ECAS2021/Program}{eCAS 2021} for \cite{scafiloci}
	\item Talk at \href{https://conf.researchr.org/program/acsos-2021/program-acsos-2021/?date=Fri%201%20Oct%202021}{ACSOS 2021 Doctoral Symposium} for \cite{research}
	\item Artifact Evaluation Committee at \href{https://conf.researchr.org/committee/acsos-2021/acsos-2021-papers-artifact-evaluation-committee}{ACSOS 2021}
\end{enumerate}
\section{Courses and School}

\begin{table}[h]
	\resizebox{\textwidth}{!}{%
	\begin{tabular}{|c|c|c|c|c|c|}
	\hline
		Professor & Course & Kind & Credits & Period & Exam \\ \hline
		Matteo Ferrara & \href{https://www.unibo.it/it/didattica/insegnamenti/insegnamento/2021/455807}{Deep Learning} & LM Course & \makecell{50 hours \\ 5 proposed credits} & \makecell{March - June \\ 2021} & To do \\ \hline
		Mirco Musolesi & \makecell{Reinforcement Learning \\ for Autonomous \\ Systems Design} & PhD Course & \makecell{16 hours \\ no credits} & \makecell{September \\ 2020} & Not done \\ \hline
		Mirco Musolesi & \makecell{Advanced Topics in \\ Reinforcement Learning} & PhD Course & \makecell{10 hours \\ no credits} & \makecell{April \\ 2021} & Not done \\ \hline
		X & \makecell{European Agent \\ Systems Summer School \\ \href{https://paginas.fe.up.pt/~easss2021/}{\textbf{EASSS 2021}}} & \makecell{PhD Summer \\ School} & \makecell{28 hours \\ 4 proposed credits} & \makecell{July \\ 2021} & \makecell{Not present \\ certificate of attendance} \\ \hline
	
	\end{tabular}
	}
\end{table}

\section{Pubblications}
\printbibliography[heading=none, check=mine]
\section{References}
\printbibliography[heading=none, check=other]

\end{document}
