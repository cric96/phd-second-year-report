\documentclass[12pt]{article}
\usepackage{mgates-letter}
\definecolor{dark_blue} {rgb}{0., 0., 0.65}
\usepackage{makecell}
\usepackage[
backend=biber,
style=alphabetic,
%citestyle=numeric,
%entrykey=false,
%annotation=false,
url=false
]{biblatex}

\usepackage{textcomp}
\usepackage{mathrsfs}  % mathscr font
\usepackage{boxedminipage}
\usepackage{rotating}
\usepackage{csquotes}
%\usepackage{natbib}
\usepackage[colorlinks, filecolor=dark_blue, urlcolor=dark_blue, linkcolor=black, citecolor=black]{hyperref}

\defbibcheck{mine}{\iffieldequalstr{annotation}{mine}{}{\skipentry}}
\defbibcheck{other}{\iffieldequalstr{annotation}{other}{}{\skipentry}}

\addbibresource{biblio.bib}

\begin{document}

\begin{center}
	{{
		\Large{
			\textsc{PhD Programme in Computer Science and Engineering \\ Cycle XXXVI}
			}
	}} 
	\rule[0.1cm]{\textwidth}{0.1mm}
	\rule[0.4cm]{\textwidth}{0.6mm}
\end{center}

\begin{center}
	{\LARGE{Engineering Cyber Physical Swarms with Aggregate Computing}} \\
	\vspace{4mm}
	{\large{PhD Year I -- Report}} 
	\vspace{4mm}
\end{center}
\vspace{8mm}
\par
\noindent
\begin{minipage}[t]{0.47\textwidth}

{\large{Commission: \\\bf
Prof. Mirko Viroli \\
Prof. Andrea Omicini \\
Prof. Matteo Ferrara} 
}
\end{minipage}
\hfill
\begin{minipage}[t]{0.47\textwidth}
	\raggedleft
	{
		\large{PhD Student: \\\bf Gianluca Aguzzi }
	}
\end{minipage}
\vspace{10mm}

{
	\raggedright
	\rule[0.1cm]{\textwidth}{0.6mm}
	\rule[0.5cm]{\textwidth}{0.1mm}
}

\newcommand{\rev}[1]{{\color{red}#1}}
\section{Overall activity}
My current research topic concerns engineering large scale multi-agent systems \rev{leveraging 
 oportunistically both of Aggregate Computing and Machine Learning approaches}.
 In particular, I am focused on \textit{Cyber Physical Swarms} (CPS), 
 a broad term that refers to a ``swarm" (like whom
 existing in nature) of computational (simple) entities in which different goals are 
 achieved collectively through a \rev{global/system-wide behavioural specification}.
%
Engineers working in this field have to deal with distributed control, 
 huge nodes count, and highly stochastic environments.

So in literature, different solutions aim at handling this complexity.
 Among the many, \textit{Aggregate Computing}~\cite{aggregatecomputing} is one of the most
 prominent approaches to ``program" collective behaviour for such kind of systems.
%
Indeed it enables the definition of self-organising collective behaviour 
 through a manipulation of a distributed data called \textit{computational field}.
 In this way, the program specification is ``collectively" declared, \rev{abstracting over
 underlying aspects -- such as network topology or communication protocol -- and making it 
 applicable to any system size.}
 %and so, theoretically, can scale up to any system size.
%
Aggregate Computing is proposed in various scenarios, 
 such as \rev{crowds of people, smart cities, and large-scale IoT~\cite{processes}.}
\rev{
In my research project, we would like to extend 
 Aggregate Computing in the field of CPS
 which can be conceived as a modern version of the notion of swarm,
 including  not  only swarm  of  robots  and  drones,
 but  also  ``swarms"  of  IoT devices and swarms of
 people with wearable devices in a real environment.
 
Towards this direction, distributed intelligence, flexible middleware, and ad-hoc
 building block need to be conceived, widening Aggregate Computing both
 at the foundational level and at the ``architectural" level.
}
In order to explore the foundational level, in this first year, 
 my activities have been centred on the integration of Aggregate Computing 
 with Machine Learning capability 
 to create even more intelligent collective behaviour.
%
Indeed, even if Aggregate Computing is already applied in various scenarios,
 the behaviour specification can lead to
 a complex fine-tuning sometimes led by domain experience.
%
Furthermore, some behaviours work properly only in specific contexts and when environmental
 changes happen the behaviour became unstable.
%
Definitively we think that the exploitation of 
 learning can lead to more complete collective behaviours, 
 improving our current state-of-the-art solutions, making Aggregate 
 Computing more and more applicable in real large scale systems.

In the first part of my work, I tried to understand what 
 kind of machine learning algorithms is most suitable for Aggregate Computing.
%
\rev{My work climaxes} with \textit{\citetitle{research}} where I explain the different suitable approach
 to enhance Aggregate Computing with Machine learning.
%
Finally, in the last period, I made concrete experiments with 
 Reinforcement Learning -- in particular Q-Learning~\cite{watkins1992q} -- 
 improving the current Aggregate Computing solution. 
 These works are not described in a publication yet, but we plan to organise them in a paper in a 
 few months

Aside to ``Machine Learning'' integration, in \textit{\citetitle{collectiveautonomy}} 
 we explore the relationship between Aggregate Computing and agents autonomy. In particular, 
 we investigate the position of Aggregate Computing 
 w.r.t individual autonomy and the collective autonomy describing how we can selectively ``adjust" it. 

\rev{Concerning} the ``architectural" level, my research activities aim at closing the \rev{gap} between
 its abstract space and its application in concrete systems. 
%
In this direction, I made mainly contributions 
 to ScaFi~\cite{scafi}\footnote{\url{https://github.com/scafi/scafi}}, 
 a research-oriented tool-chain that brings Aggregate Computing into the Scala world.
 The works consist in:

\begin{itemize}
  \item \textit{\citetitle{scafiweb}}\footnote{\url{https://github.com/scafi/scafi-web}}: 
	A playground to experiment with Aggregate Computing. 
	From a long-term perspective, this tool should be used to support our vision of ``hybrid" systems,
	where a part of the ensemble is composed of simulated entities. 
	Furthermore, ScaFi-web should be used in support for the distributed monitoring of Aggregate Computing application
  \item \textit{\citetitle{scafiloci}}: 
	An approach for deploying aggregate programs in different networks. This work have a long-term vision of separating functional aspects from non-functional aspects.
\end{itemize}

\section{Talks}
\begin{enumerate}
	\item \citefield{scafiweb}{title} @ \href{https://www.discotec.org/2021/programme}{COORDINATION 2021}
	\item \citefield{scafiloci}{title} @ \href{https://apice.unibo.it/xwiki/bin/view/ECAS2021/Program}{eCAS 2021}
	\item \citefield{research}{title} @ \href{https://conf.researchr.org/program/acsos-2021/program-acsos-2021/?date=Fri%201%20Oct%202021}{ACSOS 2021 Doctoral Symposium}
\end{enumerate}
\section{International Conference Activities}
\begin{enumerate}
	\item Artifact Evaluation Committee @ \href{https://conf.researchr.org/committee/acsos-2021/acsos-2021-papers-artifact-evaluation-committee}{ACSOS 2021}

\end{enumerate}
\section{Courses and School}

\begin{table}[h]
	\resizebox{\textwidth}{!}{%
	\begin{tabular}{|c|c|c|c|c|c|}
	\hline
		Professor & Course & Kind & Credits & Period & Exam \\ \hline
		Matteo Ferrara & \href{https://www.unibo.it/it/didattica/insegnamenti/insegnamento/2021/455807}{Deep Learning} & LM Course & \makecell{50 hours \\ 5 proposed credits} & \makecell{March - June \\ 2021} & To do \\ \hline
		Mirco Musolesi & \makecell{Reinforcement Learning \\ for Autonomous \\ Systems Design} & PhD Course & \makecell{16 hours \\ no credits} & \makecell{September \\ 2020} & Not done \\ \hline
		Mirco Musolesi & \makecell{Advanced Topics in \\ Reinforcement Learning} & PhD Course & \makecell{10 hours \\ no credits} & \makecell{April \\ 2021} & Not done \\ \hline
		X & \makecell{European Agent \\ Systems Summer School \\ \href{https://paginas.fe.up.pt/~easss2021/}{\textbf{EASSS 2021}}} & \makecell{PhD Summer \\ School} & \makecell{28 hours \\ 4 proposed credits} & \makecell{July \\ 2021} & \makecell{Not present \\ certificate of attendance} \\ \hline
	
	\end{tabular}
	}
\end{table}

\section{Papers}
\subsection{Published}
\begin{itemize}
	\item \fullcite{scafiweb}
	\\{\footnotesize\textbf{Abstract:} \citefield{scafiweb}{abstract}}
	\item \fullcite{collectiveautonomy}
	\\{\footnotesize\textbf{Abstract:} \citefield{collectiveautonomy}{abstract}}
\end{itemize}
\subsection{Accepted}
\begin{itemize}
	\item \fullcite{scafiloci}
	\\{\footnotesize\textbf{Abstract:} \citefield{scafiloci}{abstract}}
	\item \fullcite{research}
	\\{\footnotesize\textbf{Abstract:} \citefield{research}{abstract}}
\end{itemize}
\section{References}
\printbibliography[heading=none, check=other]

\end{document}
