\documentclass[11pt]{article}
\usepackage{mgates-letter}
\definecolor{dark_blue} {rgb}{0., 0., 0.65}
\usepackage{makecell}
\usepackage[
backend=biber,
style=alphabetic,
%citestyle=numeric,
%entrykey=false,
%annotation=false,
url=false
]{biblatex}

\usepackage{textcomp}
\usepackage{mathrsfs}  % mathscr font
\usepackage{boxedminipage}
\usepackage{rotating}
\usepackage{csquotes}
%\usepackage{natbib}
\usepackage[colorlinks, filecolor=dark_blue, urlcolor=dark_blue, linkcolor=black, citecolor=black]{hyperref}

\defbibcheck{mine}{\iffieldequalstr{annotation}{mine}{}{\skipentry}}
\defbibcheck{other}{\iffieldequalstr{annotation}{other}{}{\skipentry}}

\addbibresource{biblio.bib}

\begin{document}
\sloppy
\begin{center}
	{{
		\Large{
			\textsc{PhD Programme in Computer Science and Engineering \\ 
			\vspace{4mm}
			Cycle XXXVI}
			}
	}} 
	\rule[0.1cm]{\textwidth}{0.1mm}
	\rule[0.4cm]{\textwidth}{0.6mm}
\end{center}

\begin{center}
	{\LARGE{A Language-based Software Engineering Approach for Cyber-Physical Swarms}} \\
	\vspace{4mm}
	{\large{PhD Year II -- Report}} 
	\vspace{4mm}
\end{center}
\vspace{8mm}
\par
\noindent
\begin{minipage}[t]{0.47\textwidth}

{\large{Commission: \\\bf
Prof. Mirko Viroli \\
Prof. Andrea Omicini \\
Prof. Matteo Ferrara} 
}
\end{minipage}
\hfill
\begin{minipage}[t]{0.47\textwidth}
	\raggedleft
	{
		\large{PhD Student: \\\bf Gianluca Aguzzi}
	}
\end{minipage}
\vspace{10mm}

{
	\raggedright
	\rule[0.1cm]{\textwidth}{0.6mm}
	\rule[0.5cm]{\textwidth}{0.1mm}
}

\newcommand{\rev}[1]{{
	%\color{red}
	#1
	}}
\section{Background}
% \rev{
% 	My research topic concerns engineering large-scale multi-agent 
% 	systems by improving current state-of-the-art algorithms, 
% 	injecting machine learning to improve adaptivity, 
% 	and advancing current distributed platforms to close the gap to real-world systems.
% }

%  In particular, I am focused on \textit{Cyber-Physical Swarms} (CPSW), 
%  a broad term that refers to a ``swarm" (like whom
%  existing in nature) of computational (simple) entities in which different goals are 
%  achieved collectively.
%  %through a \rev{global/system-wide behavioural specification}
% %
% Engineers working in this field have to deal with distributed control, 
%  huge nodes count, and highly stochastic environments.

% So in literature, different solutions aim at handling this complexity.
%  Among the many, \textit{Aggregate Computing}~\cite{aggregatecomputing} is one of the most
%  prominent approaches to ``program" collective behaviour for such kinds of systems.
%
\rev{
In my research project, I deal with Cyber-Physical Swarm (CPSW), 
 which are highly complex systems -- conceptually inspired by natural swarms -- 
 where a collection of (simple) entities reach collective goals through self-organising behaviours.
%
Swarm robotics, ``swarms" of people (crowds) or, in general, ``swarms" of IoT devices are clearly defined instances of CPSW.

These systems typically exist in largely unpredictable environments and they have no fixed size (possibly very large).
%
Furthermore, they are composed of entities that have partial observability of the system, 
 and that influence the environment to which they belong.
 %
Moreover, these systems are collocated in the modern IT scenario, and so we have to cope with complex and layered networks.

Nowadays, researchers are trying to handle these properties by working on the construction of robust self-adaptive collective behaviour -- 
 like the one observed in a natural environment, and so failure-resistant, efficient, and effective. 
%
For years, the behaviours synthesis has been guided by natural phenomena observation that was then transposed into computer systems --- a so-called bottom-up approach. 
 However, this trend led to specific solutions that hardly scale up with application complexity.

So my research goal is to find a systematic methodology (models, techniques and algorithms) to synthesise and deploy self-organising behaviours of predictable outcomes for CPSW.
 It is a broad-spectrum goal and it concerns the improvement of both foundational aspects of the current state-of-the-art paradigms, 
 and the infrastructural level to apply research results into concrete systems. 
 Finally, for us, Machine Learning acts as an adjunct between these levels to push toward robust systems that can adapt to unforeseen situations.

About the state-of-the-art paradigms,} \textit{Aggregate Computing}~\cite{aggregatecomputing} is one of the most prominent approaches to ``program" such kinds of systems.
%
Indeed it enables the definition of self-organising collective behaviour 
 through manipulation of a distributed data structure called \textit{computational field}.
%
 In this way, the program specification is ``collectively" declared, \rev{abstracting over underlying aspects -- 
 such as network topology or communication protocol -- 
 and making it applicable to any system size.}
 %and so, theoretically, can scale up to any system size.
%
Aggregate Computing is proposed in various scenarios, 
 such as \rev{crowds of people, smart cities, and large-scale IoT~\cite{processes}.}

 \rev{
%In my research project, we would like to extend 
% Aggregate Computing in the field of CPSWs.
% To move towards this direction, distributed intelligence, flexible middlewares, and ad-hoc
%  building blocks need to be conceived, widening Aggregate Computing both
%  at the foundational level and at the ``architectural" level.

Finally, in my research project, we would like to investigate and analyse critically Aggregate Computing techniques in the field of CPSWs.
 This investigation will address multiple directions like distributed intelligence,
 flexible middlewares, and ad-hoc building blocks, 
 possibly leading to contributions both at a foundational and ``architectural" level.
}

\section{Activities}
In order to explore the foundational level, in this first year, 
 my activities have been centred on the integration of Aggregate Computing 
 with Machine Learning capability 
 to create even more intelligent collective behaviour.
%
Indeed, even if Aggregate Computing is already applied in various scenarios,
 the behaviour specification can lead to
 a complex fine-tuning sometimes led by domain experience.
%
Furthermore, some behaviours work properly only in specific contexts and when environmental
 changes happen the behaviour became unstable.
%
Definitively we think that the exploitation of 
 learning can lead to more complete collective behaviours, 
 improving our current state-of-the-art solutions, making Aggregate 
 Computing more and more applicable in real large scale systems
 and \rev{helping the automatic synthesis of self-organising behaviours.}

In the first part of my work, I tried to understand what 
 kind of machine learning algorithms is most suitable for Aggregate Computing.
%
\rev{My work climaxes} with \textit{\citetitle{research}} where I explain different suitable approaches
 to enhance Aggregate Computing with Machine learning.
%
Finally, in the last period, I made concrete experiments with 
 Reinforcement Learning -- in particular Q-Learning~\cite{watkins1992q} -- 
 improving the current Aggregate Computing solutions. 
 These works are not described in a publication yet, but we plan to organise them in a paper in a 
 few months.

Aside from ``Machine Learning'' integration, in \textit{\citetitle{collectiveautonomy}} 
 we explore the relationship between Aggregate Computing and agents autonomy. In particular, 
 we investigate the position of Aggregate Computing w.r.t individual autonomy and the collective autonomy describing how we can selectively ``adjust" it. 

\rev{Concerning} the ``architectural" level, my research activities aim at closing the \rev{gap} between
 its abstract space and its application in concrete systems. 
%
In this direction, I made mainly contributions 
 to ScaFi~\cite{scafi}\footnote{\url{https://github.com/scafi/scafi}}, 
 a research-oriented tool-chain that brings Aggregate Computing into the Scala world.
 The works consist in:

\begin{itemize}
  \item \textit{\citetitle{scafiweb}}\footnote{\url{https://github.com/scafi/scafi-web}}: 
	A playground to experiment with Aggregate Computing. 
	From a long-term perspective, this tool should be used to support our vision of ``hybrid" systems,
	where a part of the ensemble is composed of simulated entities. 
	Furthermore, ScaFi-web should be used in support of the distributed monitoring for Aggregate Computing applications.
  \item \textit{\citetitle{scafiloci}}: 
	An approach for deploying aggregate programs in different networks topologies. 
	This work has a long-term vision of separating functional aspects from non-functional aspects.
\end{itemize}
\section{Period Abroad}
In August, I started my period abroad 
 at Aarhus University under the supervision of Lukas Esterle.
%
During this period, I was mainly interested in applying Graph Neural Networks (GNNs)
 -- A neural network that works on graphs -- 
 with Aggregate Computing programs.
%
In particular, we apply GNN to forecast 
 the state of natural phenomena tracked by agents of a system, and then, 
 using Aggregate Programming, we would like to coordinate the agent in a way 
 that they could follow the phenomena of interest.
%
In doing this, we applied a Centralised Training and Decentralised Execution (CTDE) approach. 
%
In this way, the learning local features using global data, 
 and then we can ``break'' the neural architecture in each node of the system.
%
This could be done since GNN can be also understood as a local application of matrix multiplication.
%
This integration has a broader impact on the research of Aggregate Computing combined with Machine Learning. 
%
Indeed, this could be used in the work already 
 explore to extract a representation of the local state agent 
 --- that in fact we found particularly under using hand-craft solutions. 

\section{Talks}
\begin{enumerate}
	\item Towards Reinforcement Learning-based Aggregate Computing @ COORDINATION 2022
	\item Machine Learning for Aggregate Computing: a Research Roadmap @ DISCOLI 2022
	\item Addressing Collective Computations Efficiency: Towards a Platform-level Reinforcement Learning Approach @ ACSOS 2022
\end{enumerate}
\section{International Conference Activities}
\begin{enumerate}
	\item Artifact Evaluation Committee @ COORDINATION 2022
	\item Student Volunteer @ ICDCS 2022
	\item Student Volunteer @ ACSOS 2022
	\item Sub-reviewer @ COORDINATION 2022
	\item Sub-reviewer @ AAMAS 2022
	\item Sub-reviewer @ DISCOLI 2022
	\item Sub-reviewer @ ASE NIER 2022
\end{enumerate}
\section{Teaching}
\begin{enumerate}
	\item Tutor @ Padigmi di Progettazione e Sviluppo (PPS)
	\item Tutor @ Programmazione Concorrente e Distribuita (PCD)
	\item Seminar entitled ``Scala: a Cross-Platform Language'' @ PPS
	\item Seminar entitled ``Scala(e) to the large. Concurrent programming in Scala and relevant Frameworks'' @ PPS
	\item Seminar entitled ``Akka: An introduction'' @ PCD
	\item Seminar entitled ``Akka for Distributed Systems'' @ PCD
\end{enumerate}
\section{Courses and School}

\begin{table}[h]
	\resizebox{\textwidth}{!}{%
	\begin{tabular}{|c|c|c|c|c|c|}
	\hline
		Professor & Course & Kind & Credits & Period & Exam \\ \hline
		Enrico Gallinucci & From Big Data to Data Platform & PhD Course & \makecell{10 hours \\ 2 proposed credits} & \makecell{January \\ 2022} & Not done \\ \hline
		Roberto Casadei & \makecell{Engeneering Collective Intelligence} & PhD Course & \makecell{10 hours \\ 2 proposed credits} & \makecell{December \\ 2020} & Done \\ \hline
		Marco Gori & \makecell{Towards Developmental Machine Learning} & \makecell{BISS 2022 \\ Spring School} & \makecell{10 hours \\ 4 credits } & \makecell{April \\ 2022} & Done \\ \hline
		Massimo Villari & \makecell{From Cloud to Serverless through microelements} & \makecell{BISS 2022 \\ Spring School} & \makecell{10 hours \\ 4 credits } & \makecell{April \\ 2022} & Done \\ \hline
		Armir Bujari & \makecell{Industry 4.0 and the Industrial Internet of Things: \\ Challenges and Enabling Technologies} & \makecell{PhD Course} & \makecell{15 hours \\ 3 credits } & \makecell{February \\ 2022} & Done \\ \hline
		Aristides Gionis & \makecell{Opinions and conflict in social networks: \\ models, computational problems, and algorithms} & \makecell{BISS 2022 \\ Spring School} & \makecell{X} & \makecell{February \\ 2022} & Not Done \\ \hline
		X & \makecell{MOOC \\ Introduction to Complexity \\ \href{https://www.complexityexplorer.org/courses/119-introduction-to-complexity-2021}{\textbf{MOOC}}} & \makecell{MOOC} & \makecell{15 hours \\ 3 proposed credits} & \makecell{September \\ 2022} & \makecell{Done} \\ \hline
	
	\end{tabular}
	}
\end{table}

\section{Papers}
\subsection{Published}
\begin{itemize}
	\item \fullcite{swarm-clustering}
	\\{\footnotesize\textbf{Abstract:} \citefield{swarm-clustering}{abstract}}
	\item \fullcite{aguzzi2022towards}
	\\{\footnotesize\textbf{Abstract:} \citefield{aguzzi2022towards}{abstract}}
\end{itemize}
\subsection{Accepted}
\begin{itemize}
	\item \fullcite{scafi-paper}
	\\{\footnotesize\textbf{Abstract:} \citefield{scafi-paper}{abstract}}
	\item \fullcite{roadmap}
	\\{\footnotesize\textbf{Abstract:} \citefield{roadmap}{abstract}}
	\item \fullcite{engineering}
	\\{\footnotesize\textbf{Abstract:} \citefield{engineering}{abstract}}
	\item \fullcite{rl-middleware}
	\\{\footnotesize\textbf{Abstract:} \citefield{rl-middleware}{abstract}}
\end{itemize}
\subsection{In Peer Review}
\begin{itemize}
	\item \fullcite{domains}
	\\{\footnotesize\textbf{Abstract:} \citefield{domains}{abstract}}
\end{itemize}
\section{References}
\printbibliography[heading=none, check=other]

\end{document}
